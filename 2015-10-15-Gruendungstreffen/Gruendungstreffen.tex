% Disclaimer % 
% Bei Verwendung von GIT kann auf den gesamten Quellcode zugegriffen werden
% Daher NIEMALS vertauliche Notizen in den Quellcode schreiben!

%%%%%%%%%%%%%%%%%%%%Präämbel%%%%%%%%%%%%%%%%%%%%%
\documentclass[a4paper,12pt,oneside]{scrartcl} % Klasse von KOMA
\usepackage{scrpage2} % für Kopf- und Fußzeile, kommt aus KOMA
\usepackage[ngerman]{babel}
\usepackage[T1]{fontenc}
\usepackage[utf8]{inputenc}

%XeLaTeX, ermöglicht UTF und TrueType-Fonts
%\usepackage{xltxtra} % Funktionen für XeLaTeX
%\usepackage{polyglossia} % Lokalisierung
\usepackage{totpages}% Gesamtseiten
\usepackage{colortbl}% Farbe fuer Tabellen
\usepackage{xcolor}% Farbpaket
%\setdefaultlanguage{german}

% Corporate-Identity %
%\setmainfont[Mapping=tex-text]{Julianna}
\definecolor{forumsorange}{HTML}{F0910B}
\definecolor{forumsgrau}{HTML}{9C9D9F}
\blendcolors{!75}

%Seitenränder, etc.
\usepackage[]{geometry}
\geometry{a4paper, top=12.5mm, left=25mm, right=25mm, bottom=1.25cm}

\usepackage{eurosym} % immer \officialeuro benutzen, denn € fehlt in Julianna
\setlength{\parindent}{0pt}

% Kop- und Fußzeile
\pagestyle{scrheadings}
\setkomafont{pageheadfoot}{\normalfont\scriptsize}
\clearscrheadfoot %Entfernt die automatische Seitennummerierung
%Fuß
\ifoot{
\rule{\textwidth}{0.2pt}
\textbf{ML-KA} Hochschulgruppe für Maschinelles Lernen

%%%% Zentrierte Seitennummerierung am Seitenende %%%%
\begin{center}
Seite \pagemark \ von \ref{TotPages}\\
\sitzgsNR . Treffen
\end{center}
%%%%---------------------------------------------%%%%
}

%Kopf und Logo
\chead{
\begin{center}
{\LARGE \textbf{ML-KA}}\\
{{\normalsize Hochschulgruppe für Maschinelles Lernen}}
\end{center}
}

% Höhe von Kopf- und Fußzeile anpassen
\setlength{\headheight}{90pt}
\setlength{\textheight}{600pt}
\setlength{\footskip}{30mm}

\setkomafont{sectioning}{\rmfamily \bfseries} 
\renewcommand{\thesection}{TOP \arabic{section}} % TOP Ueberschrift
\newcommand*{\fs}[2]{#1\textsubscript{#2}} % Befehl für Fachsemester:  \fs{1,...n: Semesterzahl}{B, M oder MB}

%%% Uebersicht für Abstimmungsergebnisse %%%
% Es werden 5 Argumente benötigt
% SYNTAX \abstimmung{Titel der Abstimmung}{Anzahl insg.}{Dafür}{Dagegen}{Enthaltung}
\newcommand*{\abstimmung}[5]{
%\begin{center}
\begin{tabular}[c]{>{\columncolor{forumsorange}}c% 
>{\columncolor{forumsorange}}c%
>{\columncolor{forumsorange}}c%
>{\columncolor{forumsorange}}c}
\multicolumn{4}{>{\columncolor{forumsgrau}}c}{#1} \\
Stimmen insgesamt & Dafür & Dagegen & Enthaltung \\
#2 & #3 & #4 & #5 \\
\end{tabular}
%\end{center}
}
%%% ------------------------------------ %%%

%%% Uebersicht fuer Personenwahlen %%%
% Fuer <=7 Wahlvorschlaege
% SYNTAX \wahl{Wahltitel/Abstimmungstitel}{Name Kandidat 1 & Ja Stimmen & Nein Stimmen & Enthaltungen\\}...{Name Wahlsieger}
\newcommand*{\wahl}[9]{
%\begin{center}
\begin{tabular}[c]{>{\columncolor{forumsorange}}c% 
>{\columncolor{forumsorange}}c%
>{\columncolor{forumsorange}}c%
>{\columncolor{forumsorange}}c}
\multicolumn{4}{>{\columncolor{forumsgrau}}c}{#1}\\
Name & Ja & Nein & Enthaltung\\
#2
#3
#4
#5
#6
#7
#8
\multicolumn{4}{>{\columncolor{forumsgrau}}c}{Damit ist #9 gewählt.}
\end{tabular}
\end{center}
}
%%% ------------------------------ %%%

%!!!!!!!!!!!!!!!!!!!!!ÄNDERN!!!!!!!!!!!!!!!!!!!!!
%%%%%%%%%%%%%%%%%%%%%%%%%%%%%%%%%%%%%%%%%%%%%%%%%
\newcommand{\sitzgsNR}{1} % Nummer der Sitzung
\newcommand{\sitzgsLtr}{Martin Thoma} % Sitzungsleiter
\newcommand{\sitzgsPrtk}{Johannes Reiß} % Protokollant
\newcommand{\sitzgsDate}{15.10.2015}
\newcommand{\sitzgsOrt}{Donnerstag, den \sitzgsDate , 15:00 Uhr, Raum 167 Geb. 20.20} % Zeit und Ort, der Sitzung
%%%%%%%%%%%%%%%%%%%%%%%%%%%%%%%%%%%%%%%%%%%%%%%%%

%%%%%%%%%%%%%%%%%%%%%%%%%%%%%%%%%%%%%%%%%%%%%%%%%
%Wieder nicht ändern!
\begin{document}
%\begin{flushright}
% \today
%\end{flushright}
\rule{\textwidth}{1pt}

% Titelzeile
\begin{center}
 \textbf{\Large{Protokoll des \sitzgsNR . Treffen am \sitzgsDate}}\\
 Gründungstreffen - Mitgliederversammlung
\end{center}

\begin{tabular}{l l}
 \textbf{Zeit \& Ort:}	& \sitzgsOrt	\\
 \textbf{Redeleitung:}	& \sitzgsLtr	\\
 \textbf{Protokoll:}	& \sitzgsPrtk	\\
\end{tabular}
%%%%%%%%%%%%%%%%%%%%%%%%%%%%%%%%%%%%%%%%%%%%%%%%%

\rule{\textwidth}{1pt}

%%%%%%%%%%%%%%%%%%%%%%TOPS%%%%%%%%%%%%%%%%%%%%%%%
Begrüßung durch Martin Thoma und kurze Vorstellung der Tagesordnung.

\section{Name und Satzung} % Erster TOP, Nummerierung erfolgt natürlich automatisch
\subsection*{Name}
Es findet eine Diskussion des Namens statt.
Wünschenswert wäre wenn das KIT als Gründungsort mit einfließen könnte.\\

Es stehen die folgenden Namen zur Abstimmung.
\begin{itemize}
\item Machine Learning Karlsruhe
\item ML@KIT
\item ML-KIT
\item Forum Maschinelles Lernen
\end{itemize}
Machine Learning Karlsruhe wird einstimmig angenommen.\\
Weiter muss ein Kürzel für den Namen bestimmt werden. Zur Wahl stehen:
\begin{itemize}
\item ML-KIT
\item ML-KA
\item MLKA
\end{itemize}
ML-KA wird einstimmig angenommen.
\subsection*{Satzung}
Martin stellt die Satzung vor. Es findet eine Diskussion zu den einzelen Paragraphen statt.
\begin{description}
\item[§3 Abs. 4]Ergänzung um \glqq Ein Wiedereintritt ist in diesem Fall jederzeit möglich.\grqq\ 
\item[§5]Es wird ergänzt bzw. geändert: \glqq Der Vorstand wird für ein Jahr gewählt; er besteht aus drei Mitgliedern. Der Vorstand wählt aus seinem Kreis einen Vorsitzenden und einen Kassenwart. Diese Wahl ist zu protokollieren und den Mitgliedern zugänglich zu machen.\grqq\ \\
\glqq Die Mitgliederversammlung kann den Vorstand absetzen und einen Übergangsvorstand wählen. Dieser bleibt bis zur nächsten regulären Vorstandswahl im Amt."
\item[§6 Nr. 3]Ergänzung um, \glqq Außerordentliche Mitgliederversammlungen können von mindestens fünf Mitgliedern einberufen werden.\grqq
\end{description} 
Die Satzung wird mit den genannten Änderungen einstimmig angenommen.

\section{Mitgliedsanträge}
Es werden Mitgliedsanträge zum Ausfüllen ausgeteilt.
\section{Wahl des Vorstands}
Im Vorfeld haben sich bereits Kandidaten gemeldet. Es wird nach weiteren Kandidaten gefragt.\\

Es müssen drei Kandidaten gewählt werden. Als Wahlhelfer stellen sich Clemens Wolff und Thomas Hummel zur Verfügung.\\
Folgende Kandidaten stehen zur Wahl. (Ergebnisse der Abstimmung in Klammern)
\begin{itemize}
\item Marvin Teichmann (11)
\item Martin Thoma (11)
\item Marvin Schweizer (10)
\end{itemize}
Bei einer Enthaltung sind damit die drei zur Wahl stehenden Kandidaten gewählt.\\
Sie nehmen die Wahl an.

\section{Mitgliedsbeitrag}
Die Frage, ob ein Mitgliedsbeitrag erhoben werden soll wird diskutiert. Es wird über die Vor- und Nachteile eines Mitgliedsbeitrags gesprochen.\\
\emph{Edouard verlässt die Sitzung um 16:30 Uhr.}\\

Es wird in zwei Durchgängen über die Festlegung eines Mitgliedsbeitrags abgestimmt.\\
 
Wer ist dafür, dass im ersten Semester kein Mitgliedsbeitrag erhoben wird? (Ergebnisse der Abstimmung in Klammern)
\begin{itemize}
\item Ja (9)
\item Nein (1)
\item Enthaltung (0)
\end{itemize}
Damit wird im ersten Semester kein Mitgliedsbeitrag erhoben.\\

Wie viel soll der Mitgliedsbeitrag nach dem ersten Semester betragen? (Ergebnisse der Abstimmung in Klammern)
\begin{itemize}
\item 0\officialeuro\ (6)
\item 2\officialeuro\ (2)
\item 5\officialeuro\ (1)
\item Enthaltung (1)
\end{itemize}
Damit wird auch der Mitgliedsbeitrag im zweiten Semester ausgesetzt.

\section{Ankündigungen}
Das erste Treffen mit einer Einführung soll am 28.10.2015 um 19:15 Uhr stattfinden.\\
Es wurde ein erster Kontakt mit einer interessierten Firma hergestellt.\\
Marvin Teichmann kündigt eine Paper-Discussion Group über \glqq Pixelwise-Classification\grqq\ an. Interessierte können sich melden.

\bigskip\textbf{Ende der Sitzung: 16:43 Uhr}
\begin{flushright}
Karlsruhe, den \today
\end{flushright}

\cleardoublepage
\newpage
%
\renewcommand{\thesection}{Anhang \arabic{section}} \setcounter{section}{0}
\section{Anwesenheitsliste}
\begin{tabular}{l l l}
\textbf{Name, Vorname}	& \textbf{Hochschule}	&  \textbf{Mitglied?}	\\
Thoma, Martin & KIT & ja \\
Teichmann, Marvin & KIT & ja \\
Hummel, Thomas & KIT & ja \\
Reiß, Johannes & KIT & ja \\
Schweizer, Marvin & KIT & ja \\
Stumpp, Matthias & KIT & ja \\
Fuhrmann, Tino & KIT & ja \\
Haas, Benedikt & KIT & ja \\
Plappert, Matthias & KIT & ja \\
Fouché, Edouard & KIT & ja \\
Wolff, Clemens & KIT & ja \\
\end{tabular}


\end{document}
